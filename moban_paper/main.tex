% -------------------------
% Author: 杨睿
% Github: https://github.com/Yangruipis
% E-mail: yangruipis@163.com
% -------------------------
% !TeX root = main.tex
\documentclass{SUIBEthesis}
\usepackage{booktabs, lmodern}
\usepackage[comma, sort&compress,numbers, square,super]{natbib}
\usepackage[nottoc]{tocbibind}      % 将参考文献加入目录
\usepackage{amsmath}
\usepackage{tocloft}
\usepackage{amssymb}
\usepackage{diagbox} % 表格中的斜线
\usepackage{multirow} % 多行表格
\usepackage{gbt7714} % 国标参考文献
\usepackage{threeparttable} % 表格附注
\numberwithin{figure}{section}
\numberwithin{table}{section}   
\addtocontents{toc}{\protect\thispagestyle{empty}}


\begin{document}
 
\title{标题}
\author{姓名}
\studentnum{学号}
\advisor{导师}
\degree{专业}
\school{上海对外经贸大学}
\major{专业}
\finishdate{2023年11月}

\maketitle

\baselineskip=20pt % 全文20pt行距
\newpage
\begin{sloppypar}
\begin{abstract}

这里是中文摘要

\end{abstract}

\keywords{关键字}

\newpage
\begin{enabstract}

这里是英文摘要
\end{enabstract}

\enkeywords{英文关键字}


% ------------------- 目录页 -----------------------
\newpage
\tableofcontents
\thispagestyle{empty}

% --------------------------------------------------

% --------------------------------------------------正文部分
\newpage
\setcounter{page}{1} 
\pagestyle{mainpage}
\section{绪论}
正文1

\subsection{国内外研究现状}
正文1.1
\subsubsection{时间序列模型的分解与重构}
正文1.1.1

\section{测试公式}
\begin{equation}
  \left[
  \begin{array}{c}
     y_1^{(i)}(t) \\
     y_2^{(i)}(t) 
  \end{array}
  \right]
  =
  \left[
  \begin{array}{cc}
      1 & 1 \\
      1 & -1
  \end{array}
  \right]
  \left[
  \begin{array}{c}
      x \\ 
      w^(i)
  \end{array}
  \right]
  \nonumber
  \end{equation}

\subsection{测试多行公式}
\begin{align}
  &y_t = (f^{(l)}(x)_{t-1})\theta_t^{(l)}+\epsilon_t^{(l)} \label{eq:ml-dma}\\
  &\theta_t^{(l)} = \theta_{t-1}^{(l)}+\eta_t^{(l)}
\end{align}

\subsection{测试表格}
\begin{table}[htbp]
  \centering
  \caption{不同决策树算法的特征选择标准}
  \begin{tabular}{llll}
    \toprule
    \toprule
    算法    & 特征选择标准  & 是否多叉树 & 适用范围\\
    \midrule
    ID3  & 信息增益   & 多叉树 & 仅分类问题\\
    C4.5  & 信息增益熵 & 多叉树 & 仅分类问题\\
    CART  & 基尼系数/平方误差 & 二叉树 & 分类和回归问题\\
    \bottomrule
    \bottomrule
  \end{tabular}%
  \label{tab:2}%
\end{table}%


\subsection{测试带附注的表格}
\begin{table}[htbp]
  \centering
  \caption{测试带附注的表格}
  \begin{threeparttable} 
  \begin{tabular}{cccc}
    \toprule
    \toprule
    1 & 2 & 3 & 4\\
    \midrule
    1& & & \\
    2& & & \\
    3& & & \\
    4& & & \\
    \bottomrule
    \bottomrule
  \end{tabular}
  \begin{tablenotes}
    \footnotesize
    \item[1] 这里是表格附注
  \end{tablenotes}
\end{threeparttable} 

\end{table}%


\subsection{测试表格中带斜线}

\begin{table}[htbp]
  \centering
  \caption{测试表格中带斜线}
  \begin{tabular}{cccccccc}
    \toprule
    \toprule
    \diagbox{$\lambda$}{$\alpha$}& 1& 2 \\
    \midrule
    1 & & \\
    2 & & \\
    \bottomrule
    \bottomrule
  \end{tabular}%
  \label{tab:md1}%
\end{table}%



% 测试参考文献引用:


% 本文采用顺序编码制,使用国标7714格式,具体细节参见gbt7714 github地址
% 采用作者+年份,参考文献处则没有顺序
% 采用顺序编码,参考文献处则有标号
\bibliographystyle{gbt7714-numerical}

测试引用\cite{Machines2002}

测试引用\citet{Machines2002}

测试引用\citep{Machines2002}

多重引用\citet{Machines2002,RandomForests2001}


\xiaosi
\bibliography{reference}



\begin{myappendix}
\begin{python}[moreemph={[4]42},caption={RV分解}]
import numpy as np
import pandas as pd
import scipy.stats as st
\end{python}
\end{myappendix}






\begin{mythanks}

\thispagestyle{empty}
感谢我的老师和小伙伴!
\end{mythanks}

\end{sloppypar}
\end{document}


